\documentclass[
    12pt,
    a4paper,
    draft
]{article}

    %! TEX TS-program = xelatex

%===================general packages=======================
\usepackage{amsmath,amssymb,amsfonts}
\usepackage{eurosym,wasysym}
\usepackage{enumitem}
\usepackage[
	colorlinks=true,
	linkcolor=black,
	citecolor=black,
]{hyperref}
\usepackage{graphicx}
\usepackage{xcolor}
\usepackage{multicol}
\setlength{\columnsep}{1.5cm}
\usepackage[normalem]{ulem}
\usepackage{fancyhdr}
\usepackage{nicefrac}
\usepackage{lipsum}
\usepackage{emptypage}
\usepackage{tabularx}
\usepackage{comment}


%===================appearance=============================
\usepackage[
%	left=3cm,
%	right=3cm,
%	top=3cm,
%	bottom=3cm
]{geometry}

\fancypagestyle{fancy}{
\fancyhf{}
\fancyhead[LE,RO]{\slshape \rightmark}
\fancyhead[LO,RE]{\slshape \leftmark}
\fancyfoot[C]{\thepage}
\renewcommand{\headrulewidth}{0.4pt}
}


\headheight 42pt
%===================language===============================
\usepackage{fontspec}
\usepackage{polyglossia}
%\usepackage{arabicnumbers}
\setdefaultlanguage{german}
\setotherlanguage[numerals=mashriq]{arabic}

\newfontfamily\arabicfont[Script=Arabic, Mapping=arabicdigits]{ScheherazadeNew-Regular.ttf}

\usepackage[
    autostyle=true,
]{csquotes}

%===================Quran==================================
\usepackage[
    uthmani,
    nopar,
    transde,
    %bubenheim,
    wordwise,
]{quran}
\usepackage[
bubenheim
]{quran-de}

\deSetTrans{bubenheim}

%===================own commands===========================
\newcommand{\ayah}[2]{
    \begin{quote}{
            \flushright
            \textarabic{\char"FD3F\quranayah[#1][#2]\char"FD3E}

        }
        \quranayahde[#1][#2]\footnote{
            \glsn{sura} \surahname[#1] %|
            %\textarabic{
            %سورة
            %\surahname*[#1]
            %}
        }
    \end{quote}
}

\newcommand{\hadith}[4]
	{\begin{quote}
	{\flushright \textarabic{#1}

	}
	
	#3\footnote{#4 | \textarabic{#2} }
	
	\end{quote}
}

\newcommand{\ayahchunk}[4]{
    \begin{quote}{
            \flushright\textarabic{﴿\quranayah[#1][#2][#4]﴾}

        }
        \quranayahde[#1][#2][#3]\footnote{
            \glsn{sura} \surahname[#1], Teil von \glsn{aya} #2
        }
    \end{quote}
}

%===================Bibliography===========================
\usepackage[backend=biber,							% biber statt bibtex
	bibencoding=utf8,							    	% utf8 für deutsche Umlaute
	bibwarn=true,								    	% Warnung bei fehlerhafter bib-Datei
	natbib=true,								    	% natbib-Kompatibilität
	%style=numeric-comp,						    		% Zitationsstil
	style = authortitle-ibid,
%	citestyle=authoryear,
	maxcitenames=2,						    			% Max. Anzahl genannter Autoren
	sorting=nty,						    			% Numerische Zitierreihenfolge
	giveninits=true,					    			% Vornamen als Initialien
	isbn=false,							    			% keine ISBN in der Literatur
	doi=false,							    			% keine DOI in der Literatur
	url=false							    			% keine URL in der Literatur
]{biblatex}							        			% Biblatex
\DefineBibliographyStrings{ngerman}{				% "et al." statt "u.a."
   andothers = {{et\,al\adddot}},}					% "

\setcounter{biburllcpenalty}{9999}	% allows linebreaks after lower case letter for high value but below 10000
\setcounter{biburlucpenalty}{9999}	% same but after upper case letters
\setcounter{biburlnumpenalty}{10000}	% same but for numbers

\addbibresource{bibliography.bib}

%===================global settings========================
\setlength\parindent{0pt}
\renewcommand\LayoutCheckField[2]{#1#2}
\fboxsep=0pt
\setcounter{tocdepth}{3}
\setcounter{secnumdepth}{3}

%\includeonly{Fiqh}

%===================END OF PREAMBEL========================
