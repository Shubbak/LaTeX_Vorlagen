%==================TITLEPAGE=====================
%\maketitle 
\pagestyle{empty}
\newgeometry{
    inner=2cm,
	outer=2cm,
	top=2cm,
	bottom=2cm
    }
\clearpage

\vspace*{2ex}

\noindent\makebox[\textwidth]{\rule{\textwidth}{0.4pt}}%\\[10pt]
\noindent
{\huge
\begin{center}
\textbf{title}
\end{center}
}

\noindent\makebox[\textwidth]{\rule{\textwidth}{0.4pt}}

\begin{center}

\vspace{3pt}
{
\large
Vorgelegt von:
%\vspace{10pt}

\textbf{Abdulrahman  Shubbak}

}

\vspace{93pt}

{
\Large
\textbf{Abschlussarbeit}

}
\vspace{10pt}
Date

\end{center}

\vspace{10pt}
\begin{center}
{\large
Durchgeführt an der:
}
\vspace{0pt}
\begin{figure}[ht]
\begin{minipage}{0.6\textwidth}
\centering
\includegraphics[width=0.45\textwidth]{Logo_Uni-Kassel}
\end{minipage}
\begin{minipage}{0.4\textwidth}
\flushleft
\includegraphics[width=0.45\textwidth]{AGE}
\end{minipage}
\end{figure}

\vspace{0pt}
{\large
Universität Kassel

\vspace{10pt}
Fachbereich 10 -- Mathematik und Naturwissenschaften

\vspace{10pt}
Institut für Physik
\vspace{10pt}

Arbeitsgruppe Experimentalphysik IV -- Dünne Schichten und Synchrotronstrahlung 

\vspace{20pt}
Unter Anleitung von Dr.  Andreas Hans

}
\end{center}

\normalsize
\restoregeometry
%==================TABLE OF CONTENTS=============
\cleardoublepage
\thispagestyle{empty}
{\hypersetup{linkcolor=black}
\tableofcontents
}


