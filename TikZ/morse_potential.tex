\documentclass{article}
\usepackage[utf8]{inputenc}
\usepackage{siunitx}
\usepackage{tikz}
\usepackage{pgfplots}
\usepackage{xfrac}


\begin{document}

\usetikzlibrary{math}
\usetikzlibrary{intersections} 
\usetikzlibrary{calc}

\tikzmath{
    function morse(\x, \De, \a, \re, \v1, \r1) {
        return \v1 + \De * (1 - exp(-\a * (\x - (\re + \r1))))^2;
    };
    \DX = 4.52;  % = equil. En.
    \DB = 4.1;  % = equil. En.
    \rX = 0.741; % = equil. distance
    \rB = 0.732; % = equil. distance
    \a = 1.942; % spring const.
    \aB = 2.014; % spring const.
    \x1 = 0.3; % domain
    \x2 = 2;   % domain
    \r1 = 0.1; % translation of equil. distance
    \v1 = 6;    % translation of equil. energy
}


\tikzmath{\x0 = 0.7; \x3 = 0.9; \x4 = 1.1;} % fixed points

\begin{tikzpicture}
    \begin{axis}[
        title={Morse Potential for $\mathrm{H_2}$ in the $X$-State},
        xlabel={$\sfrac{r}{\si{\angstrom}}$},
        ylabel={$\sfrac{V(r)}{\si{\eV}}$},
        ylabel style={rotate=-90},
        domain=0.5:5, samples=100
        ]
        \addplot[name path = X, thick, samples=1000, domain=\x1:\x2]
            {morse(x, \DX, \a, \rX, 0, 0)};
        \addplot[name path = B, thick, red, samples=1000, domain=\x1 + \r1:\x2]
            {morse(x, \DB, \aB, \rB, \v1, \r1)};
        % \draw[->, thick] let \p1 = (X), \p2 = (B) in (\p1) -- (\p2) ;

        \draw[->, dashed] 
            (axis cs: \x0 , {morse(\x0, \DX, \a, \rX, \v1, \r1)}) --
            (axis cs: \x0 , {morse(\x0, \DX, \a, \rX, 0, 0)});
        \draw[->, dotted] 
            (axis cs: \x3 , {morse(\x3, \DX, \a, \rX, \v1, \r1)}) --
            (axis cs: \x3, {morse(\x3, \DX, \a, \rX, 0, 0)});
        \draw[->, dashdotted] 
            (axis cs: \x4 , {morse(\x4, \DX, \a, \rX, \v1, \r1)}) --
            (axis cs: \x4 , {morse(\x4, \DX, \a, \rX, 0, 0)});
    \end{axis}
\end{tikzpicture}




\end{document}


